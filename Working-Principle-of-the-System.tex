<<<<<<< HEAD
%% LyX 2.3.6.1 created this file.  For more info, see http://www.lyx.org/.
%% Do not edit unless you really know what you are doing.
\documentclass[english]{article}
\usepackage[T1]{fontenc}
\usepackage[latin9]{inputenc}
\usepackage[a4paper]{geometry}
\geometry{verbose,tmargin=2.8cm,bmargin=3cm,lmargin=3cm,rmargin=3cm}
\usepackage{fancyhdr}
\pagestyle{fancy}
\usepackage{color}
\usepackage{babel}
\usepackage{float}
\usepackage{graphicx}
\usepackage[unicode=true,pdfusetitle,
 bookmarks=true,bookmarksnumbered=false,bookmarksopen=false,
 breaklinks=false,pdfborder={0 0 0},pdfborderstyle={},backref=false,colorlinks=false]
 {hyperref}

\makeatletter
\@ifundefined{date}{}{\date{}}
%%%%%%%%%%%%%%%%%%%%%%%%%%%%%% User specified LaTeX commands.
\date{}

\makeatother

\begin{document}
\title{\textbf{Design and simulate a SCADA system for }\\
\textbf{wine production line integrated with power supply}\\
\textbf{}\\
}
\author{Pham Thai Long\\
Hoang Khac Anh\\
Nguyen Tuan Vu\\
Nguyen Trung Duong\\
\\
\textbf{University of Technology of Ho Chi Minh City}\\
\textbf{}\\
\textbf{\includegraphics[scale=0.1]{Logo-BK}}}
\date{\textit{August 2011}}

\maketitle
\smallskip{}

\begin{abstract}
This document presents the working principle of our system. Taking
into consideration every technical concern related to power supply
is far beyond our goal, as we do not have many practical experiences
in this sector. Instead, while building the system, we intend to focus
on modeling the operating procedure of two modules: first, the wine
production line and second, the power distribution and generators
blocks. In the first section, we introduce four stages of the wine
production line. The second section describes a subsystem that supplies
power for the wine production line.
\end{abstract}

\lhead{\textit{The Best Application with $Unity^{TM}$ Pro \& $Vijeo\,Citect^{TM}$
SCADA}}

\rhead{\resizebox{0.4in}{!}{\includegraphics{UnityPro-Logo.jpg}}}

\lfoot{\resizebox{1in}{!}{\includegraphics{Schneider-Electric-Logo.png}}}

\rfoot{\textit{Schneider Electric Contest 2011}}

\smallskip{}

\tableofcontents{}

\newpage{}

\section{Wine Production Line\label{sec:Wine-production-line}}

This section describes the sequence that take places in our line.
Four main stages make up the wine production line: weigh and clean,
mill and ferment, press and filtrate, bottle and package. We present
these stages in the subsections belows.

\subsection{Weigh and Clean stage}

When we press the Start button (Figure \textcolor{blue}{\ref{fig:Weigh-and-Clean-(VC)}}
and \textcolor{blue}{\ref{fig:Weigh-and-Clean-(UP)}}), Motor I starts
and drives a conveyor belt that loads raw grapes into the weighing
tank. Meanwhile, water is pumped into the water tank. Once the grapes'
weight attains the preset value, the conveyor belt stops. And, if
the water tank is full, piston Number 1 pushes the grapes from the
weighing tank to the water tank. Then Motor II does its job: cleaning.
After a preset time, Motor II stops, and the water used to clean the
grapes drains off. When the tank runs out of water, piston Number
2 shifts the cleaned grapes to the second stage of the production
line: mill and ferment (Subsection \textcolor{red}{\ref{subsec:Mill-and-Ferment}}).

$\,$

\begin{figure}[h]
\begin{centering}
\includegraphics{\string"Images/Figure 1_Weigh and Clean (Vijeo Citect)\string".eps}
\par\end{centering}
\textit{\caption{\textit{Weigh and Clean stage (Vijeo Citect).\label{fig:Weigh-and-Clean-(VC)}}}
}
\end{figure}

\begin{figure}[h]
\begin{centering}
\includegraphics[scale=0.38]{\string"Images/Figure 2_Weigh and Clean (Unity Pro)\string".eps}
\par\end{centering}
\textit{\caption{\textit{Weigh and Clean Stage (Unity Pro).\label{fig:Weigh-and-Clean-(UP)}}}
}
\end{figure}

\begin{figure}[H]
\centering{}\includegraphics[scale=0.3]{\string"Images/Figure 3_Mill and Ferment (Vijeo Citect)\string".eps}\textit{\caption{\textit{Mill and Ferment Stage (Vijeo Citect).}\label{fig:Mill-and-Ferment}}
}
\end{figure}


\subsection{Mill and Ferment stage\label{subsec:Mill-and-Ferment}}

Piston Number 2 transports the cleaned grapes to the milling tank
(Figure \textcolor{blue}{\ref{fig:Mill-and-Ferment}}). When the milling
process finishes, the milled grapes are shifted to a reservoir, where
we add ferments alcoholic and gas $SO_{2}$ to transform the grape
juice into alcohol. Using a control panel (that displays the pop-up
windows - see Figure \textcolor{blue}{\ref{fig:Popup-windows}}),
we can set the temperature and the duration of the fermenting process.
At the end of the stage, grape juice (which has became alcohol) and
the insoluble residues are carried to the pressing-filtrating tank,
where the stage of press and filtrate starts (Subsection \textcolor{red}{\ref{subsec:Press-and-Filtrate}}).

\begin{figure}[h]
\begin{centering}
\includegraphics[scale=0.4]{\string"Images/Figure 4_Temperature and Time Control\string".eps}
\par\end{centering}
\textit{\caption{\textit{Temperature Control (Left) and Time Control (Right) pop-up
windows.}\label{fig:Popup-windows}}
}
\end{figure}

\begin{figure}[H]
\begin{centering}
\includegraphics[scale=0.45]{\string"Images/Figure 5_Press and Filtrate (Vijeo Citect)\string".eps}
\par\end{centering}
\textit{\caption{\textit{Press and Filtrate Stage (Vijeo Citect).}\label{fig:Press-and-Filtrate}}
}
\end{figure}


\subsection{Press and Filtrate stage\label{subsec:Press-and-Filtrate}}

First, the pressing-filtrating tank (Figure \textcolor{blue}{\ref{fig:Press-and-Filtrate}})
squeezes the mixture of alcohol and insoluble residues to separate
them. Next, the liquid part (the wine) is filtered and pumped to the
wine reservoir, while the solid part (the marc) is removed from the
tank. Now, we arrive at the final stage of the wine production line:
bottle and package (Subsection \textcolor{red}{\ref{subsec:Bottle-and-Package}}).

\subsection{Bottle and Package stage\label{subsec:Bottle-and-Package}}

In this stage, the reservoir pours wine into each bottle (Figure \textcolor{blue}{\ref{fig:Bottle-and-Package}}).
Then, eight bottles of wine are packed every time. In each production
batch, the system manufactures three packages of wine bottle.

\begin{figure}[h]
\begin{centering}
\includegraphics[scale=0.3]{\string"Images/Figure 6_Bottle and Package (Vijeo Citect)\string".eps}
\par\end{centering}
\textit{\caption{\textit{Bottle and Package Stage (Vijeo Citect). \label{fig:Bottle-and-Package}}}
}
\end{figure}

\noindent Next, we come to the integrated power supply system. Section
\textcolor{red}{\ref{sec:Power-Supply-System} }\textcolor{black}{presents
the components making up this system.}

\section{Power Supply System\label{sec:Power-Supply-System}}

In our model, the power supply system combines two specific blocks.
The first block includes two transformers and multiple circuit breakers
(see Subsection \textcolor{red}{\ref{subsec:Transformers-and-circuit}}).
The second block comprises two generators and the subsidiary devices
to supply cooling water and fuel for these generators (see Subsection
\textcolor{red}{\ref{subsec:Generators-and-fuel}}). 

\subsection{Transformers and circuit breakers\label{subsec:Transformers-and-circuit}}

To power the factory, we load the electricity from the 15 kV grid
to two transformers, Transformer I and Transformer II, through two
main circuit breakers (CB) called SWBT1 and SWBT2. The Transformer
I has three outputs: T/O\_1, T/O\_2, T/O\_3. The Transformer II has
one output: T/O\_4 (Figure \textcolor{blue}{\ref{fig:Two-transformers}}). 

$\,$

\noindent The outputs of the transformers connect to seven CBs (1st-grade
CBs). We divide these seven CBs into three groups: A (CTA1, CTA2),
B (CTB1, CTB2, CTB3), and C (CTC1, CTC2). As we can see in Figure
\textcolor{blue}{\ref{fig:1st-grade-and-2nd-grade-CBs}}, these CBs
connect to either the transformers' outputs T/O\_1, T/O\_2, T/O\_4
or the generators' outputs G/O\_1, G/O\_2. 

\begin{figure}[H]
\begin{centering}
\includegraphics[scale=0.53]{\string"Images/Figure 7_Transformers\string".eps}
\par\end{centering}
\textit{\caption{\textit{Two transformers are wired to the electrical grid}.\label{fig:Two-transformers}}
}
\end{figure}

\noindent Underneath the first-grade CBs is the second-grade ones
which include three groups: (SW1\_1, SW1\_2, and SW1\_3), (SW2\_1,
SW2\_2, and SW2\_3), and (SW3\_1, SW3\_2, and SW3\_3). The CBs in
the first two groups connect respectively to the first-grade CBs of
group A and group B above. And, the CBs SW3\_1, SW3\_2, and SW3\_3
wire, respectively, to a transformer's output (T/O\_3 of Transformer
I), the first-grade CBs (CTC1, CTC2), and a generator's output (G/O\_3
of Generator II). 

\begin{figure}[H]
\begin{centering}
\includegraphics[scale=0.42]{\string"Images/Figure 8_First and Second Circuit Breakers\string".eps}
\par\end{centering}
\textit{\caption{\textit{1st-grade and 2nd-grade CBs distribute power from the transformers'
outputs.}\label{fig:1st-grade-and-2nd-grade-CBs}}
}
\end{figure}

\noindent Among these second-grade CBs, the group (SW 3\_1, SW 3\_2,
SW3\_3) connect subsequently to the panel of third-grade CBs that
transmit power straight to different stages of the wine production
line presented in Section \textcolor{red}{\ref{sec:Wine-production-line}}.
The CBs from SW1\_1 to SW2\_2 conduct electricity to other factory's
workshops, while SW2\_3 is used as redundancy output.

$\,$

\noindent The third-grade CBs (Figure \textcolor{blue}{\ref{fig:Third-grade-CBs}})
comprise CB F1, F2, F3, F4, and F5. As we just stated, the role of
the CBs from F1 to F4 is to turn on/off power for the wine production
process. The CB F5 is the power switch of the office building.

\begin{figure}[h]
\begin{centering}
\includegraphics[scale=0.45]{\string"Images/Figure 9_Third-grade Circuit Breakers\string".eps}
\par\end{centering}
\textit{\caption{\textit{Third-grade CBs power the wine production phases.}\label{fig:Third-grade-CBs}}
}
\end{figure}

$\,$

\noindent In the next subsection, we describe the functioning of the
block of generators and fuel \& cooling water supply. For a power
supply system operating in industrial plant, this block is indispensable.

\subsection{Generators and fuel \& cooling water supply system\label{subsec:Generators-and-fuel}}

Whenever an over-voltage (the voltage on the grid exceeds 16 kV) or
an under-voltage (the voltage on the grid drops under 14kV) incident
happens in the grid, the two transformers disconnect automatically,
and the two generators (Generator I and Generator II in Figure \textcolor{blue}{\ref{fig:Generators-and-fuel}})
operate immediately. In this way, we can ensure the power supply for
the entire factory. The unique output of Generator I (G/O\_1) is wired
to the input of the first-grade circuit breakers: CTA1, CTB2, and
CTC2. Generator II has two outputs, denoted as G/O\_2 and G/O\_3,
that connect to CTB3 and SW3\_3, respectively.

$\,$

\noindent The generator, while running, creates too much heat. Thus,
a cooling process is imperative. Five seconds after the generators
operate, a closed cycle of heat dissipation takes place. Specifically,
a pump continuously pushes the water in a tank flowing through the
generator engine, where it absorbs the heat and becomes hot water.
Then, hot water is conducted to the fan that blows the heat to cool
down it before returning to the tank. And from here, cool water is
again pumped to cool down the generator engine. 

$\,$

\noindent When the problem on the grid is fixed (the voltage goes
back within the range 14 to 16 kV), the two transformers are reconnected
to the grid automatically. Meanwhile, the two generators continue
to work for five more seconds before completely stopping. This stage,
called synchronization, is to guarantee a stable power supply for
the factory.

$\,$

\noindent During the cooling process for the generator, the water
gradually runs out. If the water volume in the water tank falls under
50 l, the water valve switches on automatically to let cool water
from an external reservoir go into the tank. When the water volume
in the tank is beyond 900 l, the valve automatically switches off.

\begin{figure}[H]
\noindent \begin{centering}
\includegraphics[scale=0.47]{\string"Images/Figure 10_Generators\string".eps}
\par\end{centering}
\textit{\caption{\textit{Generators and fuel \& cooling water supply system.\label{fig:Generators-and-fuel}}}
}
\end{figure}

$\,$

\noindent The generator consumes oil reserved in his built-in tank.
While the generator is running, the fuel level inside the tank decreases.
Once the oil volume in the tank is less than 20 l, the valve of an
external oil reservoir opens automatically to provide additional fuel
for the generator's built-in tank. When the oil volume in the tank
exceeds 450 l, the oil valve automatically closes. 

$\,$

\noindent In addition, besides the automatic mode, our system also
allows the technician to turn on/off manually the generators, the
water valves, and the oil valves.
\end{document}
=======
%% LyX 2.3.6.1 created this file.  For more info, see http://www.lyx.org/.
%% Do not edit unless you really know what you are doing.
\documentclass[english]{article}
\usepackage[T1]{fontenc}
\usepackage[latin9]{inputenc}
\usepackage[a4paper]{geometry}
\geometry{verbose,tmargin=2.8cm,bmargin=3cm,lmargin=3cm,rmargin=3cm}
\usepackage{fancyhdr}
\pagestyle{fancy}
\usepackage{color}
\usepackage{babel}
\usepackage{float}
\usepackage{graphicx}
\usepackage[unicode=true,pdfusetitle,
 bookmarks=true,bookmarksnumbered=false,bookmarksopen=false,
 breaklinks=false,pdfborder={0 0 0},pdfborderstyle={},backref=false,colorlinks=false]
 {hyperref}

\makeatletter
\@ifundefined{date}{}{\date{}}
%%%%%%%%%%%%%%%%%%%%%%%%%%%%%% User specified LaTeX commands.
\date{}

\makeatother

\begin{document}
\title{\textbf{Design and simulate a SCADA system for }\\
\textbf{wine production line integrated with power supply}\\
\textbf{}\\
}
\author{Pham Thai Long\\
Hoang Khac Anh\\
Nguyen Tuan Vu\\
Nguyen Trung Duong\\
\\
\textbf{University of Technology of Ho Chi Minh City}\\
\textbf{}\\
\textbf{\includegraphics[scale=0.1]{Logo-BK}}}
\date{\textit{August 2011}}

\maketitle
\smallskip{}

\begin{abstract}
This document presents the working principle of our system. Taking
into consideration every technical concern related to power supply
is far beyond our goal, as we do not have many practical experiences
in this sector. Instead, while building the system, we intend to focus
on modeling the operating procedure of two modules: first, the wine
production line and second, the power distribution and generators
blocks. In the first section, we introduce four stages of the wine
production line. The second section describes a subsystem that supplies
power for the wine production line.
\end{abstract}

\lhead{\textit{The Best Application with $Unity^{TM}$ Pro \& $Vijeo\,Citect^{TM}$
SCADA}}

\rhead{\resizebox{0.4in}{!}{\includegraphics{UnityPro-Logo.jpg}}}

\lfoot{\resizebox{1in}{!}{\includegraphics{Schneider-Electric-Logo.png}}}

\rfoot{\textit{Schneider Electric Contest 2011}}

\smallskip{}

\tableofcontents{}

\newpage{}

\section{Wine Production Line\label{sec:Wine-production-line}}

This section describes the sequence that take places in our line.
Four main stages make up the wine production line: weigh and clean,
mill and ferment, press and filtrate, bottle and package. We present
these stages in the subsections belows.

\subsection{Weigh and Clean stage}

When we press the Start button (Figure \textcolor{blue}{\ref{fig:Weigh-and-Clean-(VC)}}
and \textcolor{blue}{\ref{fig:Weigh-and-Clean-(UP)}}), Motor I starts
and drives a conveyor belt that loads raw grapes into the weighing
tank. Meanwhile, water is pumped into the water tank. Once the grapes'
weight attains the preset value, the conveyor belt stops. And, if
the water tank is full, piston Number 1 pushes the grapes from the
weighing tank to the water tank. Then Motor II does its job: cleaning.
After a preset time, Motor II stops, and the water used to clean the
grapes drains off. When the tank runs out of water, piston Number
2 shifts the cleaned grapes to the second stage of the production
line: mill and ferment (Subsection \textcolor{red}{\ref{subsec:Mill-and-Ferment}}).

$\,$

\begin{figure}[h]
\begin{centering}
\includegraphics{\string"Images/Figure 1_Weigh and Clean (Vijeo Citect)\string".eps}
\par\end{centering}
\textit{\caption{\textit{Weigh and Clean stage (Vijeo Citect).\label{fig:Weigh-and-Clean-(VC)}}}
}
\end{figure}

\begin{figure}[h]
\begin{centering}
\includegraphics[scale=0.38]{\string"Images/Figure 2_Weigh and Clean (Unity Pro)\string".eps}
\par\end{centering}
\textit{\caption{\textit{Weigh and Clean Stage (Unity Pro).\label{fig:Weigh-and-Clean-(UP)}}}
}
\end{figure}

\begin{figure}[H]
\centering{}\includegraphics[scale=0.3]{\string"Images/Figure 3_Mill and Ferment (Vijeo Citect)\string".eps}\textit{\caption{\textit{Mill and Ferment Stage (Vijeo Citect).}\label{fig:Mill-and-Ferment}}
}
\end{figure}


\subsection{Mill and Ferment stage\label{subsec:Mill-and-Ferment}}

Piston Number 2 transports the cleaned grapes to the milling tank
(Figure \textcolor{blue}{\ref{fig:Mill-and-Ferment}}). When the milling
process finishes, the milled grapes are shifted to a reservoir, where
we add ferments alcoholic and gas $SO_{2}$ to transform the grape
juice into alcohol. Using a control panel (that displays the pop-up
windows - see Figure \textcolor{blue}{\ref{fig:Popup-windows}}),
we can set the temperature and the duration of the fermenting process.
At the end of the stage, grape juice (which has became alcohol) and
the insoluble residues are carried to the pressing-filtrating tank,
where the stage of press and filtrate starts (Subsection \textcolor{red}{\ref{subsec:Press-and-Filtrate}}).

\begin{figure}[h]
\begin{centering}
\includegraphics[scale=0.4]{\string"Images/Figure 4_Temperature and Time Control\string".eps}
\par\end{centering}
\textit{\caption{\textit{Temperature Control (Left) and Time Control (Right) pop-up
windows.}\label{fig:Popup-windows}}
}
\end{figure}

\begin{figure}[H]
\begin{centering}
\includegraphics[scale=0.45]{\string"Images/Figure 5_Press and Filtrate (Vijeo Citect)\string".eps}
\par\end{centering}
\textit{\caption{\textit{Press and Filtrate Stage (Vijeo Citect).}\label{fig:Press-and-Filtrate}}
}
\end{figure}


\subsection{Press and Filtrate stage\label{subsec:Press-and-Filtrate}}

First, the pressing-filtrating tank (Figure \textcolor{blue}{\ref{fig:Press-and-Filtrate}})
squeezes the mixture of alcohol and insoluble residues to separate
them. Next, the liquid part (the wine) is filtered and pumped to the
wine reservoir, while the solid part (the marc) is removed from the
tank. Now, we arrive at the final stage of the wine production line:
bottle and package (Subsection \textcolor{red}{\ref{subsec:Bottle-and-Package}}).

\subsection{Bottle and Package stage\label{subsec:Bottle-and-Package}}

In this stage, the reservoir pours wine into each bottle (Figure \textcolor{blue}{\ref{fig:Bottle-and-Package}}).
Then, eight bottles of wine are packed every time. In each production
batch, the system manufactures three packages of wine bottle.

\begin{figure}[h]
\begin{centering}
\includegraphics[scale=0.3]{\string"Images/Figure 6_Bottle and Package (Vijeo Citect)\string".eps}
\par\end{centering}
\textit{\caption{\textit{Bottle and Package Stage (Vijeo Citect). \label{fig:Bottle-and-Package}}}
}
\end{figure}

\noindent Next, we come to the integrated power supply system. Section
\textcolor{red}{\ref{sec:Power-Supply-System} }\textcolor{black}{presents
the components making up this system.}

\section{Power Supply System\label{sec:Power-Supply-System}}

In our model, the power supply system combines two specific blocks.
The first block includes two transformers and multiple circuit breakers
(see Subsection \textcolor{red}{\ref{subsec:Transformers-and-circuit}}).
The second block comprises two generators and the subsidiary devices
to supply cooling water and fuel for these generators (see Subsection
\textcolor{red}{\ref{subsec:Generators-and-fuel}}). 

\subsection{Transformers and circuit breakers\label{subsec:Transformers-and-circuit}}

To power the factory, we load the electricity from the 15 kV grid
to two transformers, Transformer I and Transformer II, through two
main circuit breakers (CB) called SWBT1 and SWBT2. The Transformer
I has three outputs: T/O\_1, T/O\_2, T/O\_3. The Transformer II has
one output: T/O\_4 (Figure \textcolor{blue}{\ref{fig:Two-transformers}}). 

$\,$

\noindent The outputs of the transformers connect to seven CBs (1st-grade
CBs). We divide these seven CBs into three groups: A (CTA1, CTA2),
B (CTB1, CTB2, CTB3), and C (CTC1, CTC2). As we can see in Figure
\textcolor{blue}{\ref{fig:1st-grade-and-2nd-grade-CBs}}, these CBs
connect to either the transformers' outputs T/O\_1, T/O\_2, T/O\_4
or the generators' outputs G/O\_1, G/O\_2. 

\begin{figure}[H]
\begin{centering}
\includegraphics[scale=0.53]{\string"Images/Figure 7_Transformers\string".eps}
\par\end{centering}
\textit{\caption{\textit{Two transformers are wired to the electrical grid}.\label{fig:Two-transformers}}
}
\end{figure}

\noindent Underneath the first-grade CBs is the second-grade ones
which include three groups: (SW1\_1, SW1\_2, and SW1\_3), (SW2\_1,
SW2\_2, and SW2\_3), and (SW3\_1, SW3\_2, and SW3\_3). The CBs in
the first two groups connect respectively to the first-grade CBs of
group A and group B above. And, the CBs SW3\_1, SW3\_2, and SW3\_3
wire, respectively, to a transformer's output (T/O\_3 of Transformer
I), the first-grade CBs (CTC1, CTC2), and a generator's output (G/O\_3
of Generator II). 

\begin{figure}[H]
\begin{centering}
\includegraphics[scale=0.42]{\string"Images/Figure 8_First and Second Circuit Breakers\string".eps}
\par\end{centering}
\textit{\caption{\textit{1st-grade and 2nd-grade CBs distribute power from the transformers'
outputs.}\label{fig:1st-grade-and-2nd-grade-CBs}}
}
\end{figure}

\noindent Among these second-grade CBs, the group (SW 3\_1, SW 3\_2,
SW3\_3) connect subsequently to the panel of third-grade CBs that
transmit power straight to different stages of the wine production
line presented in Section \textcolor{red}{\ref{sec:Wine-production-line}}.
The CBs from SW1\_1 to SW2\_2 conduct electricity to other factory's
workshops, while SW2\_3 is used as redundancy output.

$\,$

\noindent The third-grade CBs (Figure \textcolor{blue}{\ref{fig:Third-grade-CBs}})
comprise CB F1, F2, F3, F4, and F5. As we just stated, the role of
the CBs from F1 to F4 is to turn on/off power for the wine production
process. The CB F5 is the power switch of the office building.

\begin{figure}[h]
\begin{centering}
\includegraphics[scale=0.45]{\string"Images/Figure 9_Third-grade Circuit Breakers\string".eps}
\par\end{centering}
\textit{\caption{\textit{Third-grade CBs power the wine production phases.}\label{fig:Third-grade-CBs}}
}
\end{figure}

$\,$

\noindent In the next subsection, we describe the functioning of the
block of generators and fuel \& cooling water supply. For a power
supply system operating in industrial plant, this block is indispensable.

\subsection{Generators and fuel \& cooling water supply system\label{subsec:Generators-and-fuel}}

Whenever an over-voltage (the voltage on the grid exceeds 16 kV) or
an under-voltage (the voltage on the grid drops under 14kV) incident
happens in the grid, the two transformers disconnect automatically,
and the two generators (Generator I and Generator II in Figure \textcolor{blue}{\ref{fig:Generators-and-fuel}})
operate immediately. In this way, we can ensure the power supply for
the entire factory. The unique output of Generator I (G/O\_1) is wired
to the input of the first-grade circuit breakers: CTA1, CTB2, and
CTC2. Generator II has two outputs, denoted as G/O\_2 and G/O\_3,
that connect to CTB3 and SW3\_3, respectively.

$\,$

\noindent The generator, while running, creates too much heat. Thus,
a cooling process is imperative. Five seconds after the generators
operate, a closed cycle of heat dissipation takes place. Specifically,
a pump continuously pushes the water in a tank flowing through the
generator engine, where it absorbs the heat and becomes hot water.
Then, hot water is conducted to the fan that blows the heat to cool
down it before returning to the tank. And from here, cool water is
again pumped to cool down the generator engine. 

$\,$

\noindent When the problem on the grid is fixed (the voltage goes
back within the range 14 to 16 kV), the two transformers are reconnected
to the grid automatically. Meanwhile, the two generators continue
to work for five more seconds before completely stopping. This stage,
called synchronization, is to guarantee a stable power supply for
the factory.

$\,$

\noindent During the cooling process for the generator, the water
gradually runs out. If the water volume in the water tank falls under
50 l, the water valve switches on automatically to let cool water
from an external reservoir go into the tank. When the water volume
in the tank is beyond 900 l, the valve automatically switches off.

\begin{figure}[H]
\noindent \begin{centering}
\includegraphics[scale=0.47]{\string"Images/Figure 10_Generators\string".eps}
\par\end{centering}
\textit{\caption{\textit{Generators and fuel \& cooling water supply system.\label{fig:Generators-and-fuel}}}
}
\end{figure}

$\,$

\noindent The generator consumes oil reserved in his built-in tank.
While the generator is running, the fuel level inside the tank decreases.
Once the oil volume in the tank is less than 20 l, the valve of an
external oil reservoir opens automatically to provide additional fuel
for the generator's built-in tank. When the oil volume in the tank
exceeds 450 l, the oil valve automatically closes. 

$\,$

\noindent In addition, besides the automatic mode, our system also
allows the technician to turn on/off manually the generators, the
water valves, and the oil valves.
\end{document}
>>>>>>> b6f56d7c0eadaeac88d82e648a07629f55acc230
